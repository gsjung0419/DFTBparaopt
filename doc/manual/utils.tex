\chapter{Utility Tools}

In the following section, some utility tools will be explained. These
tools were originally developed by Michael Gaus and were later modified by
the author.

\section{Convert \textbf{repopt}-output to skf-files}

The bash-scripts \textbf{rep2XabSpl} and \textbf{xabSpl2spl} are available
in the ``utils'' directory as well as the C++ program \textbf{ord2abSpl}
which is called by the \textbf{xabSpl2spl} script. 

\begin{description}
  \item[\is{rep2XabSpl}] Usage: \textbf{rep2XabSpl repout-output-file} \\
                         The \textbf{rep2XabSpl} extracts the Spline of 
                         repulsive potentials from the output-file and
                         writes it in separate files. The ending of
                         the files are XabSpl. 
  \item[\is{xabSpl2spl}] Usage: \textbf{xabSpl2spl XabSpl-file skf-electronic-file 1}\\ 
                         The \textbf{xabSpl2spl} script combines one XabSpl
                         file with skf-electronic-file into the final skf-file.  
  \item[\is{ord2abSpl }] called by the \textbf{xabSpl2spl} 
\end{description} 

For a short description of all options run \textbf{rep2XabSpl} or
\textbf{xabSpl2spl} without any arguments.

\textbf{Example}\\ 
{\scriptsize
\noindent\# doing the rep fitting\\
repopt rep.in $>$ rep.out \\

\noindent\# extract Spline for H-H\\
rep2XabSpl rep.out \\
mv h\_h\_.4abSpl hh.4abSpl \\

\noindent\# create the final skf-files \\
xabSpl2spl hh.4abSpl hh\_elec.skf  1 \\
} 

Under \textbf{utils} folder, a script named \textbf{combine.sh} can do all
jobs at once.

\section{Plot Repulsive Potentials}

\textbf{SplineAnsch} is a script to plot repulsive potentials and its
derivatives.  The script requires \href{http://www.gnuplot.info/}{gnuplot}
and \href{https://www.gnu.org/software/gv/}{gv} ghostscript interpreter.  

\textbf{Usage}
\begin{description}
  \item[\is{one skf file:}]  SplineAnsch -a $r_{min}$:$r_{max}$ file1.skf
  \item[\is{two skf file:}]  SplineAnsch -a $r_{min}$:$r_{max}$ -v file1.skf file2.skf
\end{description} 

Note: any files in the XabSpl or spl format can be also be used.
You can find all options by running \texttt{SplineAnsch} without arguments. 

\textbf{Example}\\ 
{\scriptsize
\noindent\# to plot new cc.skf zoom in on a range of 2.0-5.0 (a.u.).\\
SplineAnsch -a 2.0:5.0 cc.skf \\

\noindent\# to compare the new cc.skf with cc.skf from mio set.\\ 
SplineAnsch -a 2.0:5.0 -v cc\_mio.spl cc.4abSpl \\

}


