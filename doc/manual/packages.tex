\usepackage{a4}
\usepackage[latin2]{inputenc}
\usepackage{t1enc}
\usepackage{amsmath}
\usepackage{amssymb}
\usepackage{enumitem} %for my check list
\usepackage{pifont}   %for my check list
\usepackage{titling}  %to modify position of titile
\usepackage{mathptmx}
\usepackage{float}
\usepackage{array}
\usepackage[english]{babel}
\usepackage{color}
\usepackage{tabularx}
\usepackage{makeidx}
\usepackage{xfrac} % nicely formated fractions
\usepackage[colorinlistoftodos]{todonotes}   % for todonote
\usepackage{soul}                            % for todonote
\usepackage{fancyvrb}
%\usepackage[colorinlistoftodos,disable]{todonotes} %disable the note
\usepackage{listings,lstautogobble}
\lstset{
  language=Pascal,
  basicstyle=\ttfamily,
  columns=fullflexible,
  breaklines=true,
  showstringspaces=true,
  autogobble=true,
  postbreak=\raisebox{0ex}[0ex][0ex]{\color{red}$\hookrightarrow$\space}
}
\definecolor{lightmagenta}{cmyk}{0,0.333,0,0}
\usepackage[pdfpagemode={UseOutlines}, pdfstartview={Fit},
        bookmarks,bookmarksnumbered,bookmarksopen,bookmarksopenlevel=0,
        plainpages=false, pdfpagelabels,
        backref=page, breaklinks=true, colorlinks=true,
        filecolor={lightmagenta},urlcolor={blue},
        linkcolor={blue},citecolor={blue}]{hyperref}

\newcommand{\pref}[1]{\pageref{#1}}              %% pageref
\newcommand{\cb}{\{\}}                           %% curly braces
\newcommand{\is}[1]{\textsf{#1}}                 %% input style
\newcommand{\iscb}[1]{\is{#1\cb}}                %% input style
\newcommand{\isl}[2]{\hyperref[#2]{\textsf{#1}}} %% input style
\newcommand{\islcb}[2]{\hyperref[#2]{\iscb{#1}}} %% input style
\newcommand{\kw}[1]{\is{#1}\index{#1@\is{#1}}}   %% keyword (with index entry)
\newcommand{\kwl}[2]{\is{#1}\index{#1@\is{#1}}\label{#2}} 
%% keyword (with index entry and label)
\newcommand{\kwcb}[1]{\iscb{#1}\index{#1\cb@\iscb{#1}}} 
%% kword with curly braces

%%% Sections, subsections etc. as hyperreference targets.
\newcommand{\htchapter}[1]{\chapter{\kw{#1}}\label{#1}}
\newcommand{\htsection}[1]{\section{\kw{#1}}\label{#1}}
\newcommand{\htsubsection}[1]{\subsection{\kw{#1}}\label{#1}}
\newcommand{\htsubsubsection}[1]{\subsubsection{\kw{#1}}\label{#1}}
\newcommand{\htparagraph}[1]{\paragraph{\kw{#1}}\label{#1}}
\newcommand{\htsubparagraph}[1]{\subparagraph{\kw{#1}}\label{#1}}
\newcommand{\modif}[1]{\is{[#1]}}
\newcommand{\modtype}[1]{{\textrm{\textit{#1}}}}
%%\newcommand{\modtype}[1]{\textsf{[}\textit{#1}\textsf{]}\ }
%%\newcommand{\modexpl}[1]{\textsf{[#1]}\ }

%%% Sections, subsections etc. as hyperreference targets (if title is a 
%%% curly braced keyword).
\newcommand{\htcbchapter}[1]{\chapter{\kwcb{#1}}\label{#1}}
\newcommand{\htcbsection}[1]{\section{\kwcb{#1}}\label{#1}}
\newcommand{\htcbsubsection}[1]{\subsection{\kwcb{#1}}\label{#1}}
\newcommand{\htcbsubsubsection}[1]{\subsubsection{\kwcb{#1}}\label{#1}}
\newcommand{\htcbparagraph}[1]{\paragraph{\kwcb{#1}}\label{#1}}
\newcommand{\htcbsubparagraph}[1]{\subparagraph{\kwcb{#1}}\label{#1}}

%%% Table of properties
    %\begin{tabular*}{\linewidth}{|>{\sf}l>{\sf}l>{\sf}l|}
\renewcommand{\tabularxcolumn}[1]{>{\raggedright\arraybackslash}p{#1}}
\newenvironment{b4table}{
  \par
  \begin{minipage}{\linewidth}
    \begin{tabular*}{\linewidth}{| >{\sf}l@{\extracolsep{\fill}} >{\sf}l@{\extracolsep{\fill}} >{\sf}l@{\extracolsep{\fill}} >{\sf}l|}
      \hline
    }
    {
      \hline
    \end{tabular*}
  \end{minipage}
}

\newenvironment{b4tableh}{
  \begin{b4table}
    \textrm{Keyword} & \textrm{Type} & \textrm{Range} & \textrm{Default} \\
    \hline }
  {
  \end{b4table}
}
\newenvironment{b5table}{
  \par
  \begin{minipage}{\linewidth}
    \begin{tabular*}{\linewidth}{| >{\sf}l@{\extracolsep{\fill}} >{\sf}l@{\extracolsep{\fill}} >{\sf}l@{\extracolsep{\fill}} >{\sf}l@{\extracolsep{\fill}} >{\sf}l|}
      \hline
    }
    {
      \hline
    \end{tabular*}
  \end{minipage}
}

\newenvironment{b6table}{
  \par
  \begin{minipage}{\linewidth}
    \begin{tabular*}{\linewidth}{| >{\sf}l@{\extracolsep{\fill}} >{\sf}l@{\extracolsep{\fill}} >{\sf}l@{\extracolsep{\fill}} >{\sf}l@{\extracolsep{\fill}} >{\sf}l@{\extracolsep{\fill}} >{\sf}l|}
      \hline
    }
    {
      \hline
    \end{tabular*}
  \end{minipage}
}
\newenvironment{b7table}{
  \par
  \begin{minipage}{\linewidth}
    \begin{tabular*}{\linewidth}{| >{\sf}l@{\extracolsep{\fill}} >{\sf}l@{\extracolsep{\fill}} >{\sf}l@{\extracolsep{\fill}} >{\sf}l@{\extracolsep{\fill}} >{\sf}l@{\extracolsep{\fill}} >{\sf}l@{\extracolsep{\fill}} >{\sf}l|}
      \hline
    }
    {
      \hline
    \end{tabular*}
  \end{minipage}
}
\newenvironment{b8table}{
  \par
  \begin{minipage}{\linewidth}
    \begin{tabular*}{\linewidth}{| >{\sf}l@{\extracolsep{\fill}} >{\sf}l@{\extracolsep{\fill}} >{\sf}l@{\extracolsep{\fill}} >{\sf}l@{\extracolsep{\fill}} >{\sf}l@{\extracolsep{\fill}} >{\sf}l@{\extracolsep{\fill}} >{\sf}l@{\extracolsep{\fill}} >{\sf}l|}
      \hline
    }
    {
      \hline
    \end{tabular*}
  \end{minipage}
}
\newenvironment{b9table}{
  \par
  \begin{minipage}{\linewidth}
    \begin{tabular*}{\linewidth}{| >{\sf}l@{\extracolsep{\fill}} >{\sf}l@{\extracolsep{\fill}} >{\sf}l@{\extracolsep{\fill}} >{\sf}l@{\extracolsep{\fill}} >{\sf}l@{\extracolsep{\fill}} >{\sf}l@{\extracolsep{\fill}} >{\sf}l@{\extracolsep{\fill}} >{\sf}l@{\extracolsep{\fill}} >{\sf}l|}
      \hline
    }
    {
      \hline
    \end{tabular*}
  \end{minipage}
}


\newenvironment{unittable}[1]{
  \par
  \begin{minipage}{\linewidth}
    \begin{tabular*}{\linewidth}{l@{\extracolsep{\fill}}l}
      \multicolumn{2}{l}{\textbf{#1:}}\\
    }
    {
    \end{tabular*}
  \end{minipage}
}

\addtolength{\hoffset}{-1.0cm}
\addtolength{\textwidth}{2.0cm}
\addtolength{\voffset}{-1.0cm}
\addtolength{\textheight}{1.5cm}

\renewcommand{\ttdefault}{\sfdefault}

%% Inverse parskip
\newcommand{\invparskip}{\vspace*{-\parskip}}

%%%%%%%%%%%%%%%%%%%%%MYCHECKLIST%%%%%%%%%%%%%%%%%%%%%%%%%%%

\newlist{mychecklist}{itemize}{2}
\setlist[mychecklist]{label=$\square$}
\newcommand{\cmark}{\ding{51}}%
\newcommand{\xmark}{\ding{55}}%
\newcommand{\done}{\rlap{$\square$}{\raisebox{2pt}{\large\hspace{1pt}\cmark}}%
                   \hspace{-2.5pt}}
\newcommand{\wrong}{\rlap{$\square$}{\large\hspace{1pt}\xmark}}

%%%%%%%%%%%%%%%%%%%%%TODONOTE%%%%%%%%%%%%%%%%%%%%%%%%%%%
\newcommand{\ncomment}[1]{\todo{\thesection{}: #1}}
\newcommand{\icomment}[1]{\todo[inline]{#1}}

\makeatletter % change the catcode of @ to 11 
\if@todonotes@disabled
  \newcommand{\hcomment}[2]{#1}
\else
  \newcommand{\hcomment}[2]{\texthl{#1}\todo{\thesection{}: #2}}
\fi
\makeatother  % change the catcode of @ back to 11    

\newcounter{lcomment}
\newcommand{\lcomment}[2][]{%
 \refstepcounter{lcomment}{%
   \todo[color={red!100!green!33},size=\small]{\thesection{}: %
      \textbf{[\uppercase{#1}\thelcomment]:}~#2}%
 }}

\newcounter{lhcomment}
\newcommand{\lhcomment}[3][]{%
 \refstepcounter{lhcomment}{%
   \texthl{#2}~\todo[color={red!100!green!33},size=\small]{\thesection{}: %
      \textbf{[\uppercase{#1}\thelhcomment]:}~#3}%
 }}

%
%\newcounter{lcomment}
%\newcommand{\lcomment}[2][]{%
% \refstepcounter{lcomment}{\texthl{#3}%
%   \todo[color={red!100!green!33},size=\small]{%
%     \textbf{[\uppercase{#1}\themycomment]:}~#2}%
% }
%}

%\newcounter{numcomment}
%\newcommand{\numcomment}[2][]{%
% \refstepcounter{numcomment}{%
%   \todo[color={red!100!green!33},size=\small]{%
%     \textbf{[\uppercase{#1}\themycomment]:}~#2}%
% }
%}
