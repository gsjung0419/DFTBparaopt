\chapter{Manual for REPOPT}
\label{chap:repopt}

``\textbf{repopt}'' is a program to optimize DFTB repulsive potentials.  To
run the program, type: 

{\scriptsize repopt rep.inp}

where \textbf{repopt} is program name and \textbf{rep.inp} is name of the
input file. The input file is organized in ``block'' sections. Each section
begins with ``\textbf{\$blockname:}'' and end with ``\textbf{\$end:}''.
`\#' is the comment character. Everything after `\#' will be skipped.

\section{Required Inputs}

The general ``block'' format of required  sections is showed as following

\begin{b4table}
\$blockname:      &         &  & \\
  \quad keyword\_1& value\_1&  & \\ 
  \quad \dots     & \dots   &  & \\  
  \quad keyword\_n& value\_n&  & \\ 
\$end             &         &  & \\
\end{b4table}

Following input block must be presented  in all kind of running job.

\subsection{\$system:}

\begin{b4tableh}
  dftb\_version  & string  &       &  dftb+   \\
  idecompose     & integer & 1:7   &  6       \\
  ilmsfit        & integer & 1:4   &  4       \\
  nreplicate     & integer &$\geq$1&  1       \\
\end{b4tableh}

\begin{description}
  \item[\is{dftb\_version}] Executable dftb-program 
  \item[\is{idecompose   }] Select decomposition method from EIGEN library:\\
       1 => ldlt \\
       2 => partialPivLu \\
       3 => fullPivLu \\
       4 => householderQr \\
       5 => colPivHouseholderQr \\
       6 => fullPivHouseholderQr \\
       7 => completeOrthogonalDecomposition \\
                            Please check the
                            \href{http://eigen.tuxfamily.org/dox/group__TutorialLinearAlgebra.html}{website}
                            for more information. \\
  \item[\is{ilmsfit      }] Select  regression method from EIGEN library:\\
       1 => householderQr \\
       2 => colPivHouseholderQr \\
       3 => fullPivHouseholderQr \\
       4 => bdcSvd \\
                            Please check the
                            \href{http://eigen.tuxfamily.org/dox/group__LeastSquares.html}{website}
                            for more information. \\

  \item[\is{nreplicate   }] Number of the fitting will be replicated.
                            \textbf{nreplicate} should be 1. Larger than 1
                            only for testing purpose to measure the effect
                            of decomposition and regression methods on the
                            computing time. 
\end{description}

\textbf{Example:}
\begin{b4table}
  \$system:            &         &  &   \\
  \quad dftb\_version  &  dftb+  &  &   \\
  \quad idecompose     &  6      &  &   \\
  \quad ilmsfit        &  4      &  &   \\
  \quad nreplicate     &  1      &  &   \\
  \$end:               &         &  &   \\
\end{b4table}

\subsection{\$genetic\_algorithm:}

\begin{b4tableh}
  ga             & bool    &  0|1  &   1       \\
  runtest        & bool    &  0|1  &   0       \\
  score\_type    & integer & 1|2|4 &   2       \\
  read\_spline   & bool    &  0|1  &   1       \\
  popsizemax     & integer &$\geq$1&   1000    \\
  preserved\_num & integer &$\geq$0&   100     \\
  destroy\_num   & integer &$\geq$0&   10      \\
  popsizemin     & integer &$\geq$1&   2       \\
  ngen           & integer &$\geq$0&   1000    \\
  pmut           & integer &0.0:1.0&   0.02    \\
  pcross         & integer &0.0:1.0&   0.90    \\
  grid\_update   & bool    &  0|1  &   0       \\
\end{b4tableh}

\begin{description}
  \item[\is{ga             }] switch on or off genetic algorithm
  \item[\is{runtest        }] switch on or off testing job
  \item[\is{score\_type    }] set scoring function to:\\
       1 => sum of absolute deviation  \\
       2 => sum of squared  deviation  \\
       4 => sum of quartic  deviation\\
  \item[\is{read\_spline   }] how ``grid'' input file would be used:\\
       0 => only read the cutoff and count the number of knots from
                              ``grid'' input file\\
       1 => use all knots in the ``grid'' input file as initial guess\\
  \item[\is{popsizemax     }] set the initial and maximum population size for GA. 
  \item[\is{preserved\_num }] set the number of best individuals would be
                              kept from ``n-1'' to ``n'' generation.
  \item[\is{destroy\_num   }] set the number individuals be removed every
                              generation. If greater than 0, the population size 
                              will be reduced generation by generation. 
  \item[\is{popsizemin     }] set the final and minimum population size for GA. 
                              \textbf{popsizemin} is used only if \textbf{destroy\_num}$\geq$1.
  \item[\is{ngen           }] set the number of generation for GA. 
  \item[\is{pmut           }] set the mutation probability for GA. 
  \item[\is{pcross         }] set the crossover probability for GA. 
  \item[\is{grid\_update   }] How ``grid'' input file would be updated:\\ 
       0 => leave the ``grid'' input file untouched.\\
       1 => the ``grid'' input file is updated at the end of the GA optimization 
                              using the best found knots.
\end{description}
\textbf{Example:}
\begin{b4table}
  \$genetic\_algorithm:       &            &&  \\
  \quad ga                    &   1        &&  \\
  \quad runtest               &   0        &&  \\
  \quad score\_type           &   2        &&  \\
  \quad read\_spline          &   1        &&  \\
  \quad popsizemax            &   1000     &&  \\
  \quad preserved\_num        &   100      &&  \\
  \quad destroy\_num          &   10       &&  \\
  \quad popsizemin            &   2        &&  \\
  \quad ngen                  &   1000     &&  \\
  \quad pmut                  &   0.02     &&  \\
  \quad pcross                &   0.90     &&  \\
  \quad grid\_update          &   0        &&  \\
  \$end:                      &            &&  \\
\end{b4table}

\section{Optional Inputs}

The general ``block'' format of required  sections is showed as following

\begin{b4table}
\$blockname:          &              &        &             \\
  \quad entry\_name\_1& option\_1\_1 & \dots  & option\_1\_m\\ 
  \quad \dots         & \dots        & \dots  & \dots       \\  
  \quad entry\_name\_n& option\_n\_1 & \dots  & option\_n\_m\\ 
\$end                 &              &        &             \\
\end{b4table}

Following input blocks in optional, depending on the desired job.

\subsection{{\$}element\_types:}
\begin{b4table}
  String               & Real(a.u.)       & &  \\
 \hline                                       
  element\_name\_1     & atomic\_energy\_1& &  \\
  \dots                & \dots            & &  \\
  element\_name\_n     & atomic\_energy\_n& &  \\
\end{b4table}

\begin{description}
  \item[\is{element\_name }] name of fitting element, must be two lower case 
                             letters characters long. The underscore
                             character `\_' is added if element name has
                             only one character.

  \item[\is{atomic\_energy}] Atomic energy in a.u\@. for the corresponding
                             fitting element. 
\end{description}

Note: if \textbf{element\_name} is provided, \textbf{atomic\_energy}  must
be provided also\@. \textbf{atomic\_energy} (can be calculated by DFT) is
needed to fit atomization energy. If \textbf{element\_name} is not listed,
the \textbf{atomic\_energy} will be optimized. You can interpret the
meaning of \textbf{atomic\_energy} as: If \textbf{atomic\_energy} is
provided, the absolute atomization energy will be fitted (by fitting atomization
energy). If \textbf{atomic\_energy} is optimized (not provided), the
relative atomization energy (reaction energy) will be fitted (by fitting atomization energy).

\textbf{Example:}
\begin{b4table}
  \$element\_types: &            &&  \\
  \quad h\_         & -0.256789  &&  \\
  \quad c\_         & -0.456789  &&  \\
  \$end:            &            &&  \\
\end{b4table}

\subsection{{\$}repulsive\_potentials:}

\begin{b7table}
  string    &Real(\AA)& string         & Real(\AA) & integer       & integer & 0|1      \\  
  \hline
  name\_1   &min\_r   & knot-vector    & min\_step & spline order  & smooth  & negative?\\  
  \dots     & \dots   & \dots          &   \dots   &    \dots      &  \dots  & \dots    \\
  name\_n   &min\_r   & knot-vector    & min\_step & spline order  & smooth  & negative?\\  
\end{b7table}

\begin{description}
  \item[\is{name        }] name of fitting potential
  \item[\is{min\_r      }] limit the small knot. The small knot must larger than
                           or equal to shortest bond length 
                           min($R_{bond}$) - \text{min\_r}
  \item[\is{knot-vector }] name of the file containing division points in 
                           the format (in \AA):\\
                           {\scriptsize
                           knot\_1 \\
                           \dots   \\
                           knot\_n \\
                           cutoff}\\
                           The number of knot will be counted from the not-vector
  \item[\is{min\_step   }] set the smallest difference between knot
  \item[\is{spline order}] the order of spline function to be used 
                           (currently only support 4th order)
  \item[\is{smooth      }] smoothing level of that potential\\
       0 => constrain on potential energy\\
       1 => constrain on the first derivative of energy\\
       2 => constrain on the second derivative of energy\\
       3 => constrain on the third derivative of energy\\
  \item[\is{negative?   }] and allowance the potential to be attractive or not.\\
       0 => repulsive potential energy must be always positive
       1 => repulsive potential energy can be negative\\
\end{description}

\textbf{Example:}
\begin{b7table}
  \$repulsive\_potentials: &&            &         &       &     &   \\    
  \quad h\_h\_  & 0.2   & grids/hh.grdx  &   0.05  &    4  &  2  & 0 \\
  \quad c\_h\_  & 0.3   & grids/ch.grdx  &   0.05  &    4  &  2  & 1 \\
  \quad c\_c\_  & 0.3   & grids/cc.grdx  &   0.30  &    4  &  2  & 1 \\ 
  \$end         &       &                &         &       &     &   \\
\end{b7table}

\subsection{{\$}compounds:}

\begin{b7table}
  string    &Real(kcal/mol) & string  & Real      &  string   & 0|string    & integer     \\  
  \hline
  structure1& $E^{at}$      & eweight & fweight   &  dftbinp  & forceinput  & placeholder \\ 
  \dots     & \dots         & \dots   &   \dots   &    \dots  &  \dots      & \dots       \\
  structuren& $E^{at}$      & eweight & fweight   &  dftbinp  & forceinput  & placeholder \\ 
\end{b7table}

list of filenames for geometries of the fitting molecular. 

\begin{description}
  \item[\is{name        }] file name for geometry, the files need to be in 
                           xyz-format 
  \item[\is{$E^{at}$    }] reference atomization energy of the molecule. The atomization energy is defined as:
  
  \begin{equation*}
      E^{at} = -E^{tot} + \sum_{i=1}^{N_{atom}}{E_i^{atom}}
  \end{equation*}
  
  \item[\is{eweight     }] weights for energy equations
  \item[\is{fweight     }] weights for force equations
  \item[\is{dftbinp     }] input-file to run a single point energy and force 
                           calculation using the dftb 
  \item[\is{forceinput  }] for an equilibrium structure, should be a ``0'', 
                           otherwise a reference force file can be specified 
                           which is formatted as (in a.u.):\\
                           {\scriptsize\\
                           Ref\_Force-Atom\_1\_X  Ref\_Force-Atom\_1\_Y  Ref\_Force-Atom\_1\_Z \\
                           \dots \\
                           Ref\_Force-Atom\_n\_X  Ref\_Force-Atom\_n\_Y  Ref\_Force-Atom\_n\_Z \\
                           }                                                                                                                           
  \item[\is{placeholder }] for developement only, must be `0' for now
\end{description}              

\textbf{Example:}
\begin{b7table}
  \$compounds:            &       &   &   &                      &                    &   \\  
  \quad path/h2.xyz       & 109.9 & 1 & 1 &  path/dftb\_inp1.hsd &  0                 & 0 \\ 
  \quad path/ch4.xyz      & 420.1 & 1 & 1 &  path/dftb\_inp2.hsd &  0                 & 0 \\ 
  \quad path/h3cch3.xyz   & 712.0 & 1 & 1 &  path/dftb\_inp2.hsd &  0                 & 0 \\ 
  \quad path/h2\_d0.1.xyz & 000.0 & 0 & 1 &  path/dftb\_inp2.hsd &  path/hh\_d0.1.frc & 0 \\ 
  \$end                   &       &   &   &                      &                    &   \\
\end{b7table}


\subsection{{\$}definition\_reactions:}

For specifying reaction equations 

\begin{b4table}
  string           & string          &  string             &   \\  
  \hline
  abbreviation     & filename        &  dftbinp            &   \\
  \dots            & \dots           &  \dots              &   \\
  abbreviation     & filename        &  dftbinp            &   \\
\end{b4table}

\begin{description}
  \item[\is{abbreviation}] abbrev name for a geometry
  \item[\is{filename    }] file name for the geometry, the files need to be in 
                           xyz-format 
  \item[\is{dftbinp     }] input-file to run a single point energy 
                           calculation using the DFTB
\end{description}          

\textbf{Example:}
\begin{b4table}
  \$definition\_reactions:&                 &                      &   \\  
  \quad  h2               & path/h2.xyz     &  path/dftb\_inp1.hsd &   \\  
  \quad  ch4              & path/ch4.xyz    &  path/dftb\_inp2.hsd &   \\  
  \quad  h3cch3           & path/h3cch3.xyz &  path/dftb\_inp2.hsd &   \\  
  \$end                   &                 &                      &   \\
\end{b4table}

\subsection{{\$}reactions:}

\begin{b8table}
  integer     & string        &       & integer  & string        &  string & Real(kcal/mol)&  Real      \\  
  \hline
  coeff       & abbreviation  & \dots & coeff    & abbreviation  &  ->     & reactionenergy&  reaweight \\  
  \dots       & \dots         & \dots & \dots    & \dots         &  ->     & \dots         &  \dots     \\  
  coeff       & abbreviation  & \dots & coeff    & abbreviation  &  ->     & reactionenergy&  reaweight \\  
\end{b8table}

\begin{description}
  \item[\is{coeff         }] reaction coefficient\\
       if positive => reactant\\
       if negative => product 
  \item[\is{abbreviation  }] defined in the \$definition\_reactions: block
  \item[\is{reactionenergy}] reaction energy
  \item[\is{reaweight     }] weight for reaction energy equations
\end{description}          

\textbf{Example:}
\begin{b9table}
  \$reactions:   &           &    &      &     &      &    &         &      \\  
  \quad       +1 &  h3cch3   & +1 &  h2  &  -2 & ch4  & -> & -18.33  &  1.0 \\  
  \$end          &           &    &      &     &      &    &         &      \\  
\end{b9table}

\section{Output}

The output of a successful \textbf{repopt} contains:

\begin{description}
  \item[\is{scoring function      }] scoring function as a function of generation
  \item[\is{input                 }] interpreted input, a list of all distances 
                                     appearing within the reference geometries 
                                     sorted by atom type pair.
  \item[\is{technical information }] a list of number of fitting equation, number 
                                     of free variables\dots 
  \item[\is{summary of fitting    }] summary of the MSE, MUE, and RMS
  \item[\is{residual in detail    }] residuals for each equation predicted by the 
                                     fitted parameters in comparison to the 
                                     reference are listed.
  \item[\is{fitted atomic energies}] fitted atomic energies if they are optimized 
  \item[\is{repulsive potentials  }] the repulsive potentials are given in a 
                                     format of the ``Spline'' format.
\end{description}

If there is no error, \textbf{repopt} ends with a statement ``repopt normal
termination''.  Any warnings concerning the fit will appear after
\textbf{\#ga end!}.

\section{Tips}
